\documentclass{article}
\usepackage[utf8]{inputenc}

\title{Digital Methods: Learning Journal}
\author{Jonas Nebel}
\date{Autumn 2019}

\begin{document}

\maketitle

\section{Today's Date}
\subsection{Thoughts / Intentions}
\subsection{Action}
\subsection{Results}
\subsection{Final Thoughts}

\pagebreak{}

\section{31/10/2019}
\subsection{Thoughts / Intentions}

\textbf{09:30am}: Currently getting introduced to this whole overleaf thing. As a non-technical guy this all seems pretty overwhelming. I do hope everything clears up in due time. 


\textbf{10:00am}: I think I got my GitHub working. Seems like an amazing program once you learn how to use it. I succesfully created my own repository.


\textbf{10:25am}: Got both my GitHub and Overleaf working, which is a victory in itself.


\textbf{10:50pm}: After having complications with Overleaf, we are now looking at spreadsheets and how to use them to collect and sort data

\subsection{Action}

\begin{itemize}
\item Went to Overleaf to prepare the learning journal for future use
\item Entered into Introduction for Data Organization in Spreadsheets
\item Created a repository on github
\end{itemize}


Creating a repository on Github:

\begin{itemize}
\item Click on profile icon in top right corner and click ‘your repositories’
\item Click ‘new’ and entering details - named repository ‘learning-journal’
\item Selected initialize this repository with a README and created repository
\end{itemize}
 
Overleaf:
\begin{itemize}
\item Used sample template in Overleaf and edited title/abstract - recompiled and downloaded pdf for Github.

\end{itemize}

Data Carpentry Exercises:

\textit{Messy Spreadsheet}

\begin{itemize}
\item Opened data sets from previously downloaded excel spreadsheets
\item Found ‘tabs’ in excel in bottom left corner
\item Discussing messy data
\item Color used for indication
\item Multiple variables in column and row
\item Multiple values in single cells
\item Inconsistencies in null, 0 and false in each tab
\item Inconsistencies and use of asterisk
\item Blank cells are not adequate indicators
\item Multiple tabs in one spreadsheet
\end{itemize}

\textit{Metadata }


\begin{itemize}
\item Unclear data
\item There seems to be a reoccurring and limited representation of items owned. Is there a specific criteria for items to be recorded?
\item What does no to member associations exactly mean? That the members in the household are recluse or separate from the society they are living in?
\item What is the question and context for ‘affect-conflict’ - what affects the conflict?
\item Does years-liv refer to years lived in the house or community?
\end{itemize}


\textit{Formatting Problems}

\begin{itemize}
\item Use ‘blank’ for a null value - best option
\item Use _ (underscores) for spaces in between words i.e. formatting_problems
\end{itemize}\subsection{Results}

\begin{itemize}
\item Github - new repository named learning-journal with practice upload committed. 
\item Overleaf - very confused, do not have a document to work with that is in pdf format. Leaving for now, will continue journal in cloudstor and will ask for assistance in class. 
\item Data Carpentry - straight forward and informative exercises. Completed up until Formatting problems.
\end{itemize}

\subsection{Final Thoughts}

\textbf{10:55am}: Using github and overleaf will take a lot more practice and hopefully some assistance.


\section{07/11/2019 Hands-on session}

\subsection{Thoughts/Intentions}

\begin{itemize}
\item \textbf{08:00am}: Getting introduced to Data cleaning with Open Refine
\item \textbf{09:00am}: Quizzing in regex
\item \textbf{09:15am}: Now starting to work with OpenRefine. I honestly feels terrifying.

\end{itemize}

\subsection{Action}

Scoping Exercise

\begin{itemize}
\item Opening OpenRefine
\item Navigate to datacarpentry.org
\item Download files for the lesson
\item Getting data into OpenRefine
\item Transforming data with []
\item Filter data by using indicators such as >
\item Learning to save work as script
\item Import and compare scripts
\item Completed document and saved as PDF
\item Exporting our script so others can use them and replicate our work
\end{itemize}


\subsection{Final Thoughts}

\textbf{10:50am}

\begin{itemize}
\item Coding seems like a smart way to compile and understanding huge amounts of data. OpenRefine as a program showed me that big data sets is not necessarily out of scope. 
\end{itemize}

\section{14/11/2019}
\subsection{Thoughts/Intentions}
\textbf{08:15am}:  Getting introduced to Shell and Git(Bash). 


\textbf{8:40am}: Terminal, as the name is on Mac, is a really intimidating program. It feels like I am violating the very core of my PC.

\subsection{Action}

\begin{itemize}
\item Introduction to Shell
\item Navigating in different directories
\item The command: Dollarsign PWD lets you see your current directory
\item Creating copying and deleting files
\item Redirecting commands to a file with pipes
\item Introduced to Git.
\item Git is a tool to mage your source code history.

\end{itemize}





\subsection{Final Thoughts}
\textbf{10:55am} The further we got into Shell, the less intimidating it felt. It felt extremely powerful, and useful in some situtions. Still scary that you can erase your entire PC from there. Git is a strong tool to track changes in your computers files.



\section{21/11/2019}
\subsection{Thoughts/Intentions}
\textbf{08:15am}:  Have downloaded R and RStudio in preparation for this class.


\textbf{8:30am}: Really exited to learn R, seems like it have a lot of possibilities.

\subsection{Action}

\begin{itemize}
\item Setting up R and learning to nagigating it.
\item Installing different packages.
\item Learning about different arguments and assignment operators like <-
\item Assigning new values with vectors
\item Importing and getting information out of CSV files
\item Convert factors and renaming them

\subsection{Results}

\item Setting up R and learning to navigating it.
\item Able to get data into R, and manipulate it. Can make my own simple scripts.

\end{itemize}


\subsection{Final Thoughts}
\textbf{10:55am} The further we got into Shell, the less intimidating it felt. It felt extremely powerful, and useful in some situtions. Still scary that you can erase your entire PC from there. Git is a strong tool to track changes in your computers files.

\section{28/11/2019}
\subsection{Thoughts/Intentions}
\textbf{08:15am}: Exited to go deeper into R territory to see what I have to offer

\textbf{8:30am}: Especially looking forward to the graphical visualisations. 

\subsection{Action}

\begin{itemize}
\item Setting up R and learning to nagigating it.
\item Data manipulation with dplyr and tidyr 
\item Both accessible with /library(tidyverse) package 
\item Filtering rows and columns for specific information 
\item Selecting rows/columns and filtering at the same time with pipes <-  
\item Mutating variables to find ratio of values in two columns 
\item Using ggplot to plot the relationship of two variables into a data visuali-sation 
\item Learning to do all different kind of plots
\item Multiple variables in the same plot  

\subsection{Results}
\item I am now able to manipulate rows and columns
\item I am able to do a visual presentation of my data
\item Had a hard time getting the multiple variables in one plot to work

\end{itemize}

\subsection{Final Thoughts}
\textbf{10:55am} The deeper we get into R, the more comfortable I get with scripting. At the same time, I also feel like there is so much more to learn. I defnitely have to practive the multiple variable integration in one plot.


\section{28/11/2019}
\subsection{Thoughts}
\textbf{06:15am}: Currently looking around the internet for inspiration to what I should write my final paper on. I stumbled across this article: https://au.dk/univers/nyhed/artikel/fedme-er-ogsaa-et-socialt-problem/ which claims that obesity is not just a biomedical problem, but also a social one. The society, according to Anders Lindelof, promotes obesity. Having struggled with overweight most of my life, this struck something. I remembered reading this article in high-school: https://faktalink.dk/titelliste/overvaegtige-boern-og-unge that states that obesity amongst children has more than triplled over the last 30 years. It was a this point I decided I wanted to do something on the historical delopment in child obesity.


\section{01/12/2019}
\subsection{Thoughts}
\textbf{07:15am}: Still wondering what direction to take my paper in. Going to www.ourworldindata.org too see if I can get some inspiration.

\textbf{07:45am}: Found research and data on childhood obesity which I can use for a multi-country comparison.

\textbf{09:20am}: Now that both the society and the the biomedicin has been pointed at as sinners for childhood obesity, I wondered whether there could be an economic factor, seeing as childhood obesity still seems to be a problem, which would block for the efforts made to improve other sectors in preventing this issue. More specifically, I would like to investigate whether a country's wealth over time have an effect on child obesity.

\textbf{15:15am}: Found data on www.ourworldindata.org that shows the development in GDP per capita over time, which I have decided I want to use for my research.


\section{08/12/2019}
\subsection{Thoughts/intentions}
\textbf{06:15am}: Finally got around to start working on my script

\textbf{06:25am}: Downloaded both the CSV files containing data on childhood obesity and GDP per capita.

\textbf{06:30am}: Trying to get as much work done today as possible, since my son is hospitalized.

\subsection{Action}

\begin{itemize}
\item Created a project in R
\item Made two scripts in R
\item read a different CSV file in each script
\item The following procedures was replicated in both scrips
\item Started by assigning the data to a new value/variable that is easier to work with
\item Extracted data from the six countries I chose to analyze and assigned a a new value to each country, by filtering them
\item Inspected them to see that they all had equal amounts of comparable data
\item I then wanted to collect the data from the 6 countries in one variable. I encountered my first problem here
\item ERROR: I started by seperating the countries I wanted to use with commas.
\item SOLUTION: I went back to the carpentry exercises and learn I had to use | to seperate them use the column name every time a new country was to be added.
\item The GDP data did not go from 1990-2016 per default like the data with obese children, so here I had to filter out the years I wanted to use. First i tried to show years from 1990-2016 only wiht the =< command. This worked, however the dataset still had observations for every year, and I needed observations from every 5th year.
\item This was done by doing that same command that collected the countries in one value, using the Year Column instead, and what years I wanted extracted.

\subsection{Results}
\item A new dataset containing information about childhood obesity for the six countries in question
\item A new dataset containing information about GDP per capita for the 6 countries in question

\end{itemize}

\subsection{Final Thoughts}
\textbf{21:30am} Coding and making a script take a lot longer than initially realized. It does however really ease the process of large quantities of data.


\section{26/12/2019}
\subsection{Thoughts/intentions}
\textbf{07:00am}: Want to finalize the script

\textbf{07:05am}: Make a graphical illustration with ggplot containing the timeline of all 6 countries

\textbf{06:30am}: Do a correlation test

\subsection{Action}

\begin{itemize}
\item read up on the datacarpentry about ggplot
\item Trying to portray my ggplot as a timeseries with lines
\item ERROR: Would not let me make a ggplot using the line command.
\item SOLUTION: After a quick conversation with Antonio Rivero he made me realise that I needed to change the absurdly long Column name, since it was filled with expressions that means something different in R-language.
\item After changing my Column names to something more sensible, I tried making the ggplot again.
\item ERROR: Still would not work as I wanted it to.
\item SOLUTION: I asked my good friend Kristian to take a look at it, since he has experience in another code language, Python. He told me to try the "smooth" function and adding color to my country colum. This worked like a charm!
\item This was done in both scripts.
\item Now, having created a graphical data visualisation, I wanted to see whether there was any correlation between the two or not.
\item I did now now how to make a correlation or even if it was possible between scripts.
\item I found this youtube video: youtube.com/watch?v=xsL4yLBNyDg which explained how to do it.
\item The test showed an overwhelming correlation, which, after all, is very hard to generalize due to the small sample size.

\subsection{Results}
\item A graphical data visualisation showing the development in a timeline from 1990-2016 for all 6 countries concerning childhood obesity.
\item A graphical data visualisation showing the development in a timeline from 1990-2016 for all 6 countries concerning GDP per capita.
\item A quite clear correlation between the two variables, which suggest that GDP per capita have quite the impact on childhood obesity. Even though this is hard to generalize, it does at least bring awareness to this correlation, in deciding whether or not more research should be done on this.

\end{itemize}

\subsection{Final Thoughts}
\textbf{23:30am} Being finished with the scripts is really satisfying. It has really opened my eyes up for how much you can achieve by having a little finesse when it comes to coding. This has definitely been an eyeopening experience for me. I am pretty sure I will expand on this knowledge in the future.

\end{document}


